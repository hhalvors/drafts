\documentclass[12pt]{article}
\author{Hans Halvorson}
\title{Niels Bohr on causality}
\date{\today}

\usepackage{doi}
\usepackage{amssymb}
\usepackage[backend=biber,natbib=true,style=authoryear]{biblatex}
\addbibresource{master.bib}

% Define a new command \afsnit to start a new section
\newcounter{afsnitcounter} % Define a counter for the sections
\newcommand{\afsnit}{%
    \stepcounter{afsnitcounter}% Increment the counter
    \par\vspace{\baselineskip}% Ensure vertical space
    \noindent\S\ \arabic{afsnitcounter}.~~% Section header
}

\begin{document}

\maketitle

\afsnit To Do: Go through NBFS: årsag ... and conservation
principles. bevarings... 

\afsnit To read: Hartry Field, John Norton, Stanford Encyclopedia,
Schlick, Die Kausalität in der gegenwärtigen Physik; Hans Reichenbach
"Die Kausalstruktur der Welt" (Sitzungsbericht der bayer. Akademie der
Wissenschaften, Mathematisch-physikalische Klasse, 1925, Seite 133),
Ned Hall

\url{https://seop.illc.uva.nl/entries/causation-physics/}

Fenton-Glynn, Causation

David Lewis, Causation; Sara Bernstein, Lewis' View Causation

\url{https://plato.stanford.edu/entries/causation-counterfactual/}

\url{https://plato.stanford.edu/entries/causation-metaphysics/}

The meeting with the positivists?

Richard Dawid article in Sebastian Lutz book

Is causation a genuine relation?  Peter Menzies. In book for
D.H. Mellor, Real Metaphysics 

\afsnit While I don't know about \emph{all} human languages, in the
Indo-European languages, the notion of cause and effect plays a
central role. For example, it has been established beyond a reasonable
doubt that smoking \underline{causes} cancer. In contrast, I didn't
\underline{cause} the Danish national soccer team to win the European
cup in 1992 by fervently wishing that they would do so.

As with so many other things, there are two extreme views that one can
take about causality. The first extreme view --- which would seem to
validate common sense --- is that the cause-effect relation is part of
the fundamental fabric of reality; and, as such, this relation is of
central concern to science. In fact, one might go so far as to say
that it's one of the primary goals of science to uncover the causal
structure of the world \citep[see][]{salmon1984}.

The second extreme view --- championed by the empiricist David Hume,
and later by Bertrand Russell --- is that:
\begin{quote}
  The law of causality, I believe, like much that passes muster among
  philosophers, is a relic of a bygone age, surviving, like the
  monarchy, only because it is erroneously supposed to do no
  harm. \citep[p 12]{russell1913} \end{quote}

\afsnit I would assume that Russell's causal nihilism was shared by
the logical positivists. But I need to check on that.

\afsnit Among contemporary (analytic) philosophers, there are roughly
three kinds of views about causality. First, there are some few
contemporary philosophers who still maintain Russell's causal nihilism
[[cite?]]. Second, among the non-nihilists, i.e.\ those who think that
causality is a real relation, there are those who take it to be
fundamental, and those who take it to be reducible to other
notions. This last view is especially interesting, because it promises
to take the ``problematic'' concept of causality and show that it is
kosher by deriving it from other non-problematic concepts. [[David
Lewis?]]

\afsnit von Wright?

\afsnit Jim Woodward? (Judea Pearl?)

\afsnit Causality as a princple of rationality. The principle of
sufficient reason.


\afsnit Causality and freedom of the will.

There are (at least) two sources of pressure against a ``thick'' view
of the causal relation. The first source of pressure comes, as we have
seen, from empiricist scruples about this notion. In short: how could
we ever know if two events stand in the causal relation? The second
source of pressure comes from our understanding of ourselves as free
agents. In short: if every event is causally determined, then how
could we possibly have free will?

Many philosophers (and scientists) throughout history have opted for a
thick view of causality, and have believed that free will must then be
abandoned as illusory. [[Schopenhauer, Freedom of the Will?  Spinoza??
Baron d'Holbach: In his work "System of Nature". Dennett]]

Kant, of course, thought that he could have the best of both
worlds. He thought that the law of causality holds for phenomena, but
that there is freedom in the noumenal realm. Whether Kant's view on
this issue is coherent is questionable, or at least, has been
questioned by many philosophers.

One interesting thing to keep in mind about Niels Bohr is that he is
going to repeatedly say that the law of causality does \emph{not} have
to be abandoned --- although it does have to be ``generalized'' in the
principle of complementarity. (For more on that, see section ??)
However, he is also clearly committed to the freedom of the will, or,
at the very least, to the ``freedom of the experimenter''.


\afsnit Psycho-physical parallelism.

There is a popular view that can be traced back at least as far as
Leibniz (1646--1716). According to this view, there are two worlds
that run in complete parallel: the physical world and the
psychological. 




\afsnit Immanuel Kant believed that causal nihilism would entail that
science is impossible. However, he was convinced that science was
possible, so he looked for a way to block Hume's argument for the
conclusion that we never could know whether one thing causes
another. Kant argues contra Hume that experience is possible for us
only because our the phenomenal world is governed by the law of
causality. Hence, we can know in advance of experience that things
will arrange themselves into cause-effect relations. 

\afsnit The neo-Kantians made a business of trying to show that Kant's
ideas were consistent with new developments in science. They had tried
that already with the development of non-Euclidean geometry. And then
with the development of quantum mechanics, they attempted to argue
that ``quantum jumps'' did not undermine Kant's causal apriorism. The
two names that come first to mind here are Ernst Cassirer and Grete
Hermann.

\afsnit H{\o}ffding on causation. TO DO: track down what Høffding says
about causality, and perhaps, in particular, what he says about Kant's
view.

Årsagsbegrebet



\afsnit Now back to contemporary views about causality. The reason I
come back here is because analytic philosophers have put so much
effort into clarifying and categorizing the various different views
that one could take. I turn now to what is called the ``conserved
quantities'' view of causality, which had a small cadre of supporters
in the 1980s and 90s. The question is whether the conserved quantities
view sits perhaps better with the intuitions that physicists of the
early 20th century had about causal principles --- and especially with
the intuitions that Bohr had.


\afsnit Chronology.

Considering giving up momentum conservation 

\afsnit Bohr has two articles that are ostensibly devoted to the
question of causality in physics: ``causality and complementarity''
(1937), ``the causality problem in modern physics'', and ``on the
notions of causality and complementarity'' (1948).

\afsnit Causation and spacetime continuity. (Schrodinger?)

\afsnit Differential equations. Schrodinger equation versus collapse

\afsnit Discontinuity. Acausal? Does Bohr use the word acausal?

\afsnit ``causal interpretation'' 

\printbibliography 

\end{document}

%%% Local Variables:
%%% mode: latex
%%% TeX-master: t
%%% End:
