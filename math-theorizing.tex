\documentclass[12pt,fleqn]{article}
\title{What is a mathematical theory?}
\author{Hans Halvorson}
\date{\today}
\usepackage{doi}
\usepackage{amssymb}
\usepackage[backend=biber,natbib=true,style=authoryear]{biblatex}
\addbibresource{master.bib}
\begin{document}

\maketitle

%% careful about wild wild west

%% what is the content learned in a course / contained in a textbook?

%% suppes set-theoretic predicates

%% GTR relations between models 

When I reflect on my experience learning the mathematical theory of
groups \citep[see][]{hungerford}, it strikes me that there was an
ambiguity about which theory we were learning. In the earliest
lessons, we derived consequences from the axiomatic theory of
groups. For example, formulating the theory in signature
$\Sigma _0=\{ \circ ,^{-1},e\}$, with axioms such as associativity and
$\forall x(x\circ x^{-1}=e=x^{-1}\circ x)$, we quickly derived:
\[ \begin{array}{ll}
     \forall x(x\circ x=x\to x=e ) ,\\
     \forall x((x^{-1})^{-1}=x), \\
     \forall x,y((x\circ y)^{-1}=y^{-1}\circ x^{-1}) ,\end{array}
\]
each of which is a $\Sigma _0$-sentence, i.e.\ a sentence in the
first-order language of group theory. That is, these results are of
the form $T\vdash\phi$, where $T$ is the first-order theory of groups.

It might come as a surprise to a non-mathematician that a typical
textbook of group theory quickly switches to a different language with
different inference rules, and it derives a very different kind of
result. In short, most group-theoretic theorems are not results in the
first-order theory of groups, but results about the structure of
groups. For example, picking an exercise at random from the early
pages of \citep{hungerford}:
\begin{quote} (*) If $G$ is a cyclic group of order $n$ and $k|n$,
  then $G$ has exactly one subgroup of order $k$. \end{quote} This
result is not a sentence in the first-order theory of groups, and not
even in a higher-order theory based on $\Sigma _0$. Indeed, (*)
quantifies over groups and makes use of predicates such as ``$x$ is
cyclic'' that apply to groups. Which language do these predicates come
from? And which axioms are we allowed to use to prove (*)?

For clarity's sake, let's use ``group theory'' (with small ``g'') to
denote the standard first-order theory, and let's use ``Group theory''
(with big ``G'') to denote whatever the theory is in which results
such as (*) are stated and derived. To anyone who has studied even
undergraduate algebra, it will be clear that a typical course spends
much more time talking about the structure of groups (i.e.\ deriving
results in Group theory) that it does deriving results in group
theory.

The move between group theory and Group theory is hardly noticeable to
the student learning abstract mathematics. While the language and
inference rules of group theory are stated completely explicitly at
the beginning of the textbook, the language and inference rules of
Group theory are implicit, and are reinforced by following the example
of the professor or the author of the textbook. Nonetheless, the fact
that there really are \underline{two} theories in play poses a bit of
a puzzle for the philosopher who wants to understand the nature of
mathematical theories, and the epistemic commitments they involve.

To put the puzzle into question form:
\begin{quote} What is Group theory? \end{quote} Or even more
basically:
\begin{quote} In a typical textbook of g/Group theory, what is the
  language in which the theorems are stated, and what are the axioms
  from which the results are derived? \end{quote} The first, and most
obvious, proposal is that Group theory is just the theory of certain
structures that live in the set-theoretic universe. In fact, I agree
completely with the spirit of that proposal --- and as a practical way
of thinking, it is precise enough. But I'm still bothered by the fact
that Group theory isn't clearly articulated in the way that group
theory is.\footnote{If one really doesn't see the problem I'm getting
  at, then try to imagine that you are set the task of writing an
  automated theorem prover (or verifier) for Group theory. What is the
  permitted syntax in these proofs, and what are the inference rules?}

I think the intuitive idea here has to be the following:
\begin{quote}
  ($\dagger$) The predicates and relation symbols of (big ``G'') Group
  theory are definable in the language of set theory, and the results
  of Group theory are those theorems of set theory that are formulated
  in this extended language. \end{quote} As stated, ($\dagger$) would
make Group theory into a definitional extension of ZF set theory ---
and that can't be right.\footnote{I'm going to assume that ZF set
  theory is the relevant foundational theory. I don't think anything I
  say will depend on that.} In fact, if we assume --- as is typical
--- that a definitional extension of a theory is equivalent to the
original theory, then ($\dagger$) would say that Group theory and ZF
set theory are the same theory. That does too much violence to
mathematical practice. The Group theorist has a domain of study, and
it is not the same as that of the set theorist.

%% obviously it can be "reduced to" set theory. But what is the
%% relation between the theories?

It seems then that we need to modify ($\dagger$) with some kind of
domain restriction: Group theory should be \emph{about} the structure
of groups and not about arbitrary sets. Can we say that Group theory
is the theory in the extended language of set theory that applies to
groups, i.e.\ those things that satisfy the set-theoretic predicate
for being a group? But it's not at all clear how we could make sense
of this notion of ``applying to''. After all, any theorem $\phi$ of
set theory is equivalent to
\[ \phi\wedge \forall x(\mathsf{Grp}(x)\to x=x) , \] which seems to
apply to groups. One could try other, more sophisticated, ways of 

To be slightly more precise, we could actually take the $\mathsf{Grp}$
predicate to range over functional relations of type $G\times G\to G$
(thus identifying the group with the function
$\circ ^G:G\times G\to G$). For example, we could say that $x$ is a
``group'' just in case there is a set $G$ such that $x$ is an element
of $P(G\times G\times G)$ that has the relevant properties. We would
just need then to verify that all the properties are definable in the
language of set theory --- and assuredly they are.  We could then go
on to use set-theory to define the notion of a homomorphism between
groups, and to define concepts such as subgroups. The proposal then is
to \emph{extend} the language $\{ \in \}$ of ZF set theory by the
addition of these new predicate and relation symbols, e.g.\
$\mathsf{Grp}$, $\mathsf{SubGrp}$.

Thus, we have three signatures in play:
\begin{itemize} \item $\Sigma _0= \{ \circ \}$ for group theory.
\item $\Sigma _1=\{\in \}$ for set theory.
\item $\Sigma _2=\{\in ,\mathsf{Grp},\dots \}$ for Group theory.
\end{itemize}

Consider the theory $Z$ in $\Sigma ^+$ 



Claim: There is a translation of first-order group theory into the
theory of models of first-order group theory. In particular, 

Let's look at a simpler example.

Now here is what I want to get straight on: in general, to define a
notion in a theory $S$ is the same thing as interpreting some other
theory $T$ into $S$. For example, to define the inverse function
$^{-1}$ in group theory with signature $\{ \circ ,e\}$ is to show that
group theory with signature $\{ \circ ,e,^{-1} \}$ can be interpreted
into the former theory. More precisely, if we introduce the relation
$\theta$ by 
\[ \theta (x,y) \leftrightarrow x\circ y = e ,\] then it can be shown
that $\theta$ is functional.

But defining group-theoretic notions in set theory seems to be of a
different nature. The first disanalogy is that a group is (obviously)
not just a kind of set; it is a set plus some ``structure'', which in
this case, is just other sets and relations between them. If we were
going to try for a rigorous definition, we could start by defining a
predicate $\mathsf{func}(x,x,x)$ that picks out the elements of the
power set $P(G\times G\times G)$ that are functional
relations. ... the relevant predicate would look something like:
\begin{quote} $\mathsf{Grp}(x,y)$ iff $y$ is a functional relation on
  $\mathsf{prod}(x,x,x)$ that obeys the relevant axioms. \end{quote}


***

There are, of course, some obvious relations between group theory and
Group theory. In particular, each $\Sigma$-sentence $\phi$ uniquely
determines a corresponding elementary class (viz.\ all groups that
satisfy $\phi$), which is an obvious case of a predicate of Group
theory. Thus, sentences of group theory give rise to predicates of
Group theory.


Question: With the resources of set-theory we can define

Question: Is there sense to be made of the idea of a result being
provable in all models of ZF set theory?

Question: Is there a sense in which we are interpreting group theory
into set theory?

\section{Generalization}

I used group theory and Group theory as a running example, but the
fact is that large swaths of contemporary mathematics face this same
issue. In abstract algebra, there are also the theories of rings,
fields, modules, vector spaces, etc. Similarly, in point-set topology
there is a (second order) language that we use to talk about
individual spaces. E.g.\ we can show that the ... Nonetheless, the
more interesting results are about the class of all topological spaces
and the structure that class has.

\section{From mathematics to physics}

While the focus of this inquiry has been pure mathematics, it was
motivated in some part by the application of mathematics in
physics. Mature theories in physics are typically associated with
rigorous theories of pure mathematics --- e.g.\ Einstein's general
theory of relativity is associated with the theory of Lorentzian
manifolds, and quantum mechanics is associated with the theory of
Hilbert sapces (or even with $C^*$-algebras). However, philosophers of
science have struggled with understanding the relationship between the
theories of pure mathematics and the theories of physics.

In the early days of axiomatics, it seems that the picture was roughly
the following: suppose, accepting an idealization if necessary, that a
physical theory (such as Einstein's special theory of relativity)
employs a mathematical theory that can be axiomatized in first-order
logic, e.g.\ a theory $T$ in signature $\Sigma$. Then some, but
perhaps not all, of the symbols in $\Sigma$ can be given a direct
physical interpretation, and some of the consequences of $T$ can
thereby be given direct physical meaning. 

As is well documented, the original syntactic view of theories was a
failure. But we are now in a good position to see yet another reason
why it is inadequate: the syntactic view of theories has restricted
itself to the predicates from the signature $\Sigma$ (and those that
can be defined from them), when even mathematicians would avail
themselves of many more predicates. For example, imagine if we told
group theorists that they could only use predicates from the signature
$\Sigma$ of first-order group theory. Then there would be no
homomorphism theorem, no classification of finite groups, etc. We
would have to excise 90\% of the results from a typical textbook on
group theory!

So it's clear that the syntactic view of theories was based on an
overly restrictive view of the language of mathematical theories.

The semantic view of theories tried, by relaxing these assumptions, to
achieve a more realistic acccount of (rigorous) scientific
reasoning. However, we can now see that it goes too far to the
opposite extreme.

Advocates of the semantic view of theories typically assume that we
are able to pick out classes of mathematical objects via some
combination of axioms (first and higher order), or set-theoretic
predicates, or perhaps even by intellectual intuition. I will grant
that in most cases, we do in fact succeed at picking out categories
such as ``Hilbert spaces'' or ``Lorentzian manifolds'' or ``simply
connected Lie groups''.  But, and this is the key point, we need to
say true things about these structures if we are going to use them to
say true things about the world. And saying true things about these
structures requires that we make a decision about which language to
use, and which inferences to permit outselves. To say that the world
is isomorphic to a Hilbert space means nothing if one doesn't have a
sense of the criteria for isomorphism, i.e.\ of which structures need
to be preserved.

We cannot simply say that the world is isomorphic to a Hilbert
spacethe syntactic view now gets its revenge: But do we really know
But the question then is which facts are true about the relevant class
of structures --- and that once again involves both the question of
the language in which these facts are formulated.

, and the The implicit assumption here was
that there is a determinate set of ``facts'' about Hilbert spaces that
are potentially relevant for talking about how Hilbert spaces
represent the physical world. But this is precisely the problem we
faced with Group theory: if ZF set theory is consistent, then there is
a category $\mathbf{Grp}$ of groups living inside it. But what are the
facts about this category?  Similarly, there is a category
$\mathbf{Hilb}$ of Hilbert spaces living inside $U$, but what are the
facts about this category?

assumed that it is unproblematic taken a rather casual approach taken
a rather naive and blase vie60s and the advente There are two extremes
to which one could go in how to use mathematics to formulate a
physical theory. First, one could look to the analogy with (little
``g'') group theory and take the physical theory to be restricted to
the language whose sentences receive truth values in the various
models. So, for example, if we had a synthetic axiomatization $T$ of
spacetime in signature $\Sigma$, then this view would only permit
$\Sigma$-sentences to be used in formulating the physical theory. We
wouldn't be permitted, e.g., to talk about automorphisms of models, or
the embedding of one model in another.

Clearly, this first view seems far too stringent. Physicists regularly
talk about properties of models and relationships between them.

Going to the opposite extreme, one could take a kind of semantic view
of mathematical theories where it's assumed that there is a background
universe $U$ of sets, and $T$ is just some class of structures inside
$U$.

\printbibliography 

\end{document}

%%% Local Variables:
%%% mode: latex
%%% TeX-master: t
%%% End:
