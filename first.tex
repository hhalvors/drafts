\documentclass[12pt]{article}
\title{Methodologies that put something first}
\author{Hans Halvorson}
\date{\today}
\begin{document}

\maketitle

TO DO: Maudlin on canonical presentation

Metaphysics first? Mathematics first? Language first? The simple
answer is \emph{none of the above}. But I should explain.

The question of what to put first has implications for a wide range of
philosophical issues. But I will focus the inquiry around the
foundations of quantum mechanics. I am still hung up on the question
of what quantum physics means for us (i.e.\ what the philosophical
takeaway of such a strange theory is), and I'll take any help I can
get.

There are two groups of philosophers who are prepared to offer me
help. Each group has a core doctrine; and interestingly enough, these
doctrines share a common syntactic form: ``m\dots\ first''. One group
says ``metaphysics first'' and the other group says ``mathematics
first''. Each group offers convincing motivations for its doctrine;
and each doctrine picks out a unique acceptable understanding of the
lesson of quantum physics.\footnote{I must qualify. While
  ``metaphysics first'' is usually thought to lead uniquely to Bohmian
  mechanics, there are recent arguments to the effect that
  ``metaphysics first'' might rather lead to wavefunction realism.} It
behooves us, then, to look into the credentials of these ``m\dots\
first'' doctrines.

\section{Mathematics first}

It's an easier task to describe the ``mathematics first'' doctrine, as
that has already been done quite clearly by its proponent David
Wallace \citep{wallace2022}. In fact, Wallace's paper has the subtitle
``mathematics-first approaches to physics and metaphysics'' and he
indicates clearly that he believes this to be the right approach. If
one is familiar with Wallace's other work, and its beautiful
systematicity, it will also be clear that the mathematics-first
approach entails that the Everett interpretation is the unique correct
understanding of quantum mechanics.

\section{Metaphysics first}

%% Devitt

%% Fichte 

Ok, let me come clean that no philosopher of physics has literally
said ``metaphysics first,'' at least not that I know of. For me to say
that they subscribe to the metaphysics first doctrine is a bit of an
extrapolation. Nonetheless, I am confident that the case can be made
that the invocation of metaphysics first is precisely what lies behind
the arguments for Bohmian mechanics.





\end{document}

%%% Local Variables:
%%% mode: latex
%%% TeX-master: t
%%% End:
